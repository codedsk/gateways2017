
%% bare_conf.tex
%% V1.4b
%% 2015/08/26
%% by Michael Shell
%% See:
%% http://www.michaelshell.org/
%% for current contact information.
%%
%% This is a skeleton file demonstrating the use of IEEEtran.cls
%% (requires IEEEtran.cls version 1.8b or later) with an IEEE
%% conference paper.
%%
%% Support sites:
%% http://www.michaelshell.org/tex/ieeetran/
%% http://www.ctan.org/pkg/ieeetran
%% and
%% http://www.ieee.org/

%%*************************************************************************
%% Legal Notice:
%% This code is offered as-is without any warranty either expressed or
%% implied; without even the implied warranty of MERCHANTABILITY or
%% FITNESS FOR A PARTICULAR PURPOSE! 
%% User assumes all risk.
%% In no event shall the IEEE or any contributor to this code be liable for
%% any damages or losses, including, but not limited to, incidental,
%% consequential, or any other damages, resulting from the use or misuse
%% of any information contained here.
%%
%% All comments are the opinions of their respective authors and are not
%% necessarily endorsed by the IEEE.
%%
%% This work is distributed under the LaTeX Project Public License (LPPL)
%% ( http://www.latex-project.org/ ) version 1.3, and may be freely used,
%% distributed and modified. A copy of the LPPL, version 1.3, is included
%% in the base LaTeX documentation of all distributions of LaTeX released
%% 2003/12/01 or later.
%% Retain all contribution notices and credits.
%% ** Modified files should be clearly indicated as such, including  **
%% ** renaming them and changing author support contact information. **
%%*************************************************************************


% *** Authors should verify (and, if needed, correct) their LaTeX system  ***
% *** with the testflow diagnostic prior to trusting their LaTeX platform ***
% *** with production work. The IEEE's font choices and paper sizes can   ***
% *** trigger bugs that do not appear when using other class files.       ***
% The testflow support page is at:
% http://www.michaelshell.org/tex/testflow/



\documentclass[conference]{../sty/IEEEtran}

% Some very useful LaTeX packages include:
% (uncomment the ones you want to load)


% *** MISC UTILITY PACKAGES ***
%
%\usepackage{ifpdf}
% Heiko Oberdiek's ifpdf.sty is very useful if you need conditional
% compilation based on whether the output is pdf or dvi.
% usage:
% \ifpdf
%   % pdf code
% \else
%   % dvi code
% \fi
% The latest version of ifpdf.sty can be obtained from:
% http://www.ctan.org/pkg/ifpdf
% Also, note that IEEEtran.cls V1.7 and later provides a builtin
% \ifCLASSINFOpdf conditional that works the same way.
% When switching from latex to pdflatex and vice-versa, the compiler may
% have to be run twice to clear warning/error messages.






% *** CITATION PACKAGES ***
%
%\usepackage{cite}
% cite.sty was written by Donald Arseneau
% V1.6 and later of IEEEtran pre-defines the format of the cite.sty package
% \cite{} output to follow that of the IEEE. Loading the cite package will
% result in citation numbers being automatically sorted and properly
% "compressed/ranged". e.g., [1], [9], [2], [7], [5], [6] without using
% cite.sty will become [1], [2], [5]--[7], [9] using cite.sty. cite.sty's
% \cite will automatically add leading space, if needed. Use cite.sty's
% noadjust option (cite.sty V3.8 and later) if you want to turn this off
% such as if a citation ever needs to be enclosed in parenthesis.
% cite.sty is already installed on most LaTeX systems. Be sure and use
% version 5.0 (2009-03-20) and later if using hyperref.sty.
% The latest version can be obtained at:
% http://www.ctan.org/pkg/cite
% The documentation is contained in the cite.sty file itself.






% *** GRAPHICS RELATED PACKAGES ***
%
\ifCLASSINFOpdf
  % \usepackage[pdftex]{graphicx}
  % declare the path(s) where your graphic files are
  % \graphicspath{{../pdf/}{../jpeg/}}
  % and their extensions so you won't have to specify these with
  % every instance of \includegraphics
  % \DeclareGraphicsExtensions{.pdf,.jpeg,.png}
\else
  % or other class option (dvipsone, dvipdf, if not using dvips). graphicx
  % will default to the driver specified in the system graphics.cfg if no
  % driver is specified.
  % \usepackage[dvips]{graphicx}
  % declare the path(s) where your graphic files are
  % \graphicspath{{../eps/}}
  % and their extensions so you won't have to specify these with
  % every instance of \includegraphics
  % \DeclareGraphicsExtensions{.eps}
\fi
% graphicx was written by David Carlisle and Sebastian Rahtz. It is
% required if you want graphics, photos, etc. graphicx.sty is already
% installed on most LaTeX systems. The latest version and documentation
% can be obtained at: 
% http://www.ctan.org/pkg/graphicx
% Another good source of documentation is "Using Imported Graphics in
% LaTeX2e" by Keith Reckdahl which can be found at:
% http://www.ctan.org/pkg/epslatex
%
% latex, and pdflatex in dvi mode, support graphics in encapsulated
% postscript (.eps) format. pdflatex in pdf mode supports graphics
% in .pdf, .jpeg, .png and .mps (metapost) formats. Users should ensure
% that all non-photo figures use a vector format (.eps, .pdf, .mps) and
% not a bitmapped formats (.jpeg, .png). The IEEE frowns on bitmapped formats
% which can result in "jaggedy"/blurry rendering of lines and letters as
% well as large increases in file sizes.
%
% You can find documentation about the pdfTeX application at:
% http://www.tug.org/applications/pdftex





% *** MATH PACKAGES ***
%
%\usepackage{amsmath}
% A popular package from the American Mathematical Society that provides
% many useful and powerful commands for dealing with mathematics.
%
% Note that the amsmath package sets \interdisplaylinepenalty to 10000
% thus preventing page breaks from occurring within multiline equations. Use:
%\interdisplaylinepenalty=2500
% after loading amsmath to restore such page breaks as IEEEtran.cls normally
% does. amsmath.sty is already installed on most LaTeX systems. The latest
% version and documentation can be obtained at:
% http://www.ctan.org/pkg/amsmath





% *** SPECIALIZED LIST PACKAGES ***
%
%\usepackage{algorithmic}
% algorithmic.sty was written by Peter Williams and Rogerio Brito.
% This package provides an algorithmic environment fo describing algorithms.
% You can use the algorithmic environment in-text or within a figure
% environment to provide for a floating algorithm. Do NOT use the algorithm
% floating environment provided by algorithm.sty (by the same authors) or
% algorithm2e.sty (by Christophe Fiorio) as the IEEE does not use dedicated
% algorithm float types and packages that provide these will not provide
% correct IEEE style captions. The latest version and documentation of
% algorithmic.sty can be obtained at:
% http://www.ctan.org/pkg/algorithms
% Also of interest may be the (relatively newer and more customizable)
% algorithmicx.sty package by Szasz Janos:
% http://www.ctan.org/pkg/algorithmicx




% *** ALIGNMENT PACKAGES ***
%
%\usepackage{array}
% Frank Mittelbach's and David Carlisle's array.sty patches and improves
% the standard LaTeX2e array and tabular environments to provide better
% appearance and additional user controls. As the default LaTeX2e table
% generation code is lacking to the point of almost being broken with
% respect to the quality of the end results, all users are strongly
% advised to use an enhanced (at the very least that provided by array.sty)
% set of table tools. array.sty is already installed on most systems. The
% latest version and documentation can be obtained at:
% http://www.ctan.org/pkg/array


% IEEEtran contains the IEEEeqnarray family of commands that can be used to
% generate multiline equations as well as matrices, tables, etc., of high
% quality.




% *** SUBFIGURE PACKAGES ***
%\ifCLASSOPTIONcompsoc
%  \usepackage[caption=false,font=normalsize,labelfont=sf,textfont=sf]{subfig}
%\else
%  \usepackage[caption=false,font=footnotesize]{subfig}
%\fi
% subfig.sty, written by Steven Douglas Cochran, is the modern replacement
% for subfigure.sty, the latter of which is no longer maintained and is
% incompatible with some LaTeX packages including fixltx2e. However,
% subfig.sty requires and automatically loads Axel Sommerfeldt's caption.sty
% which will override IEEEtran.cls' handling of captions and this will result
% in non-IEEE style figure/table captions. To prevent this problem, be sure
% and invoke subfig.sty's "caption=false" package option (available since
% subfig.sty version 1.3, 2005/06/28) as this is will preserve IEEEtran.cls
% handling of captions.
% Note that the Computer Society format requires a larger sans serif font
% than the serif footnote size font used in traditional IEEE formatting
% and thus the need to invoke different subfig.sty package options depending
% on whether compsoc mode has been enabled.
%
% The latest version and documentation of subfig.sty can be obtained at:
% http://www.ctan.org/pkg/subfig




% *** FLOAT PACKAGES ***
%
%\usepackage{fixltx2e}
% fixltx2e, the successor to the earlier fix2col.sty, was written by
% Frank Mittelbach and David Carlisle. This package corrects a few problems
% in the LaTeX2e kernel, the most notable of which is that in current
% LaTeX2e releases, the ordering of single and double column floats is not
% guaranteed to be preserved. Thus, an unpatched LaTeX2e can allow a
% single column figure to be placed prior to an earlier double column
% figure.
% Be aware that LaTeX2e kernels dated 2015 and later have fixltx2e.sty's
% corrections already built into the system in which case a warning will
% be issued if an attempt is made to load fixltx2e.sty as it is no longer
% needed.
% The latest version and documentation can be found at:
% http://www.ctan.org/pkg/fixltx2e


%\usepackage{stfloats}
% stfloats.sty was written by Sigitas Tolusis. This package gives LaTeX2e
% the ability to do double column floats at the bottom of the page as well
% as the top. (e.g., "\begin{figure*}[!b]" is not normally possible in
% LaTeX2e). It also provides a command:
%\fnbelowfloat
% to enable the placement of footnotes below bottom floats (the standard
% LaTeX2e kernel puts them above bottom floats). This is an invasive package
% which rewrites many portions of the LaTeX2e float routines. It may not work
% with other packages that modify the LaTeX2e float routines. The latest
% version and documentation can be obtained at:
% http://www.ctan.org/pkg/stfloats
% Do not use the stfloats baselinefloat ability as the IEEE does not allow
% \baselineskip to stretch. Authors submitting work to the IEEE should note
% that the IEEE rarely uses double column equations and that authors should try
% to avoid such use. Do not be tempted to use the cuted.sty or midfloat.sty
% packages (also by Sigitas Tolusis) as the IEEE does not format its papers in
% such ways.
% Do not attempt to use stfloats with fixltx2e as they are incompatible.
% Instead, use Morten Hogholm'a dblfloatfix which combines the features
% of both fixltx2e and stfloats:
%
% \usepackage{dblfloatfix}
% The latest version can be found at:
% http://www.ctan.org/pkg/dblfloatfix




% *** PDF, URL AND HYPERLINK PACKAGES ***
%
%\usepackage{url}
% url.sty was written by Donald Arseneau. It provides better support for
% handling and breaking URLs. url.sty is already installed on most LaTeX
% systems. The latest version and documentation can be obtained at:
% http://www.ctan.org/pkg/url
% Basically, \url{my_url_here}.




% *** Do not adjust lengths that control margins, column widths, etc. ***
% *** Do not use packages that alter fonts (such as pslatex).         ***
% There should be no need to do such things with IEEEtran.cls V1.6 and later.
% (Unless specifically asked to do so by the journal or conference you plan
% to submit to, of course. )


% correct bad hyphenation here
%\hyphenation{op-tical net-works semi-conduc-tor}


\begin{document}
%
% paper title
% Titles are generally capitalized except for words such as a, an, and, as,
% at, but, by, for, in, nor, of, on, or, the, to and up, which are usually
% not capitalized unless they are the first or last word of the title.
% Linebreaks \\ can be used within to get better formatting as desired.
% Do not put math or special symbols in the title.
\title{Short Paper: cvfHUB: A Gateway Supporting Microsoft Windows in Engineering Workflows}


% author names and affiliations
% use a multiple column layout for up to three different
% affiliations
\author{\IEEEauthorblockN{David Benham}
\IEEEauthorblockA{HUBzero \\
Purdue University\\
West Lafayette, Indiana 47907\\
Email: dbenham@purdue.edu}}


% make the title area
\maketitle

% As a general rule, do not put math, special symbols or citations
% in the abstract
\begin{abstract}

The HUBzero\textsuperscript{\textregistered}  platform enables scientists to
build off one another's research by providing a robust collaborative platform
to allow them to share the software they created and utilized for their work.
Customized tools, code, and data was no longer forgotten after a research
effort was complete, HUBzero enabled researchers to offer their tools for
anyone to use and modify so they could continue to collaborate long after the
initial project was over.

Early versions of HUBzero focused on providing researchers a Linux based
environment to host their software. Our group realized there was growing
interest in HUBzero hosting commercial, licensed software, which often requires
Microsoft Windows, so we begin to investigate how we could integrate Microsoft
WIndows into our platform.  A key partner, IACMI (what does this stand for
actually) joined the effort.  In this paper, we discuss the architecture of the
solution and learnings along the way.

\end{abstract}

% no keywords


\IEEEpeerreviewmaketitle


\section{Introduction}
% no \IEEEPARstart

nanoHUB \cite{TECHCON:Kapadia96} pioneered scientific cloud computing beginning
in 1996. �Since its inception, nanoHUB has undergone several evolutions in the
manner in which it delivers scientific simulation tools over the web.
Ultimately this resulted in having developed a Linux solution from the ground
up to host simulation tools and allow users to easily put graphical interfaces
on those tools \cite{ETechNano:Lundstrom06} \cite{Rappture:McLennan04}. �The
solution for Linux was a PHP driven system that supported remote rendering of a
Linux desktop in a web browser using VNC. �In its first release, this involved
a Java applet running in the web browser. �As technologies continued to evolve,
Java encountered industry-wide security concerns and became less supported by
browser creators, and more importantly, restricted by many of the organizations
to which our users belong. �A second iteration of the technology was released
that utilizes HTML5 rather than the Java client, circumventing these
limitations.

During this same time of evolution, the infrastructure that supports nanoHUB
was also spun out as a separate entity called HUBzero with the intent of
supporting many other science gateways with the same infrastructure that drove
nanoHUB. �As more scientific communities adopted HUBzero, the support team for
HUBzero evolved the infrastructure to adapt to some of the unique needs of
those communities. �One community in particular that has become a significant
HUBzero user is in the area of virtual composite materials design and
manufacture. �This community is further from the basic science and closer to
the application of materials in the commercial marketplace than many other
HUBzero gateways. �Therefore, a significant requirement of this community was
to host Microsoft Windows based tools, both academically and commercially
developed.

To satisfy this need, the HUBzero team considered applying our own expertise to
extend our existing platform into the Windows domain. However, hosting Windows
applications and development workspaces in the cloud is significantly different
than its Linux counterpart. �The highly customized approach taken with our
LInux based system was not a viable approach as we ventured into the world of
Microsoft WIndows because the operating system and many tools were not open
source. So we began to investigate commercial off the shelf hardware.

Next we will describe the nature of the engineering workflows in cvfHUB and
motivation for the solution, several technologies tried in fulfilling this
need, and the lessons learned in providing this capability.


\section{Composite Virtual Factory HUB (cvfHUB)}

cvfHUB is focused on applying materials design principles to real world
applications in the automotive industry.  Specifically, the goal of cvfHUB is
to allow design of materials virtually, before ever manufacturing them
physically. This capability will allow thousands of possibilities to be tested
virtually before investing significant funds into manufacturing and physically
testing only the most promising options.  Typically these tasks involve
multi-objective optimization in the face of constraints such as optimizing the
weight of a given part but maintaining durability and performance criteria
within certain bounds expected to be encountered in deployment.  The workflows
involved in such part optimizations can include basic composite material
design, analysis of how the material will perform during molding and forming
processes, cutting processes, curing processes, finishing processes, and under
expected operating conditions.  Such an analysis involves the combination of
academic and commercial codes, and a significant amount of �``translation''
code in between that interpret outputs of one tool and reformation of those
outputs into inputs that a subsequent tool can accept.  Once constructed, such
a workflow may be executed hundreds or thousands of times during the design of
a given part to perform the optimization.  cvfHUB is intended to host these
workflows, along with collaboration tools for distributed engineering teams as
they undergo design activities.


\section{Solutions for Hosting Windows}

Three different technologies were evaluated and used in providing Windows
support: Amazon AppStream, Amazon AppStream v2, and Citrix.  Below these three
are contrasted with their respective capabilities and matched against the
requirements of cvfHUB.


\subsection{Amazon AppStream}

The first attempt to offer Windows tools inside our HUBzero platform involved
leveraging Amazon's AppStream service. The AppStream service was originally
designed as a cloud service that offered video game developers a way to offer
online game demos without requiring their prospective customers to download and
install large pieces of software on local clients. Utilizing AppStream,
software could be pre installed and run in on demand virtual machines running
on the Amazon Web Services platform.

AWS AppStream integration was completed in a very short timeframe, and it
initially appeared to offer an excellent user experience. The largest concern
with HUBzero's WIndows integration in the early stages was to develop as
seamless an experience as possible for end users, and the need for a highly
responsive UI and video streaming support were critical. Other benefits of
AppStream included on-demand environments that greatly helped us reduce costs
over building infrastructure, as well as a robust web browser plug-in for us
client side to display application streams to end users. However, as users
began to work with this solution, it turned out AppStream's lack of support for
shared file systems, and the level of integration with the local client did not
allow sharing clipboard text or files from local filesystems. These limitations
proved extremely difficult to overcome, and the AppStream virtual environments
themselves did not allow users to customize the environment, so we could not
overcome these limitations.

\subsection{Amazon AppStream version 2}

During our integration of AppStream, we were in contact with the software
engineers at AWS and they helped us address some of our early concerns. AWS
engineers informed us AppStream v2 was being developed. We were able to get
early access to some of the design documentation and the new feature sets they
were proposing for the next version of AppStream appeared to address some of
our concerns, so we contemplated using the next version of AppStream. However,
the AWS engineering team could not give us a firm date on the release of
AppStream version 2, and a tight deadline on our Windows integration effort
required us to abandon investigating this possibility further. We do plan on
revisiting AppStream version 2 and evaluating it for incorporation into HUBzero
at a later time.

\subsection{Amazon AppStream version 2}

Citrix is a market leader in providing remote access Microsoft Windows
software. Other groups at Purdue University were using Citrix, and early
discussions with them indicated we had some in-house experience with Citrix
technologies we could potentially leverage in our integration as we designed a
way to offer Microsoft Windows applications to our end users. Citrix based
solutions appeared to easily meet all of our design criteria, however, we
hesitated pursuing this solution at first due to the extreme learning curve
required to implement and manage a Citrix infrastructure.

As we continued to evaluate Citrix's XenApp and XenDestkop, the level of
client interoperability became a clear advantage. Citrix had spent the last
several decades designing utilities to make remote application access client
tools seamlessly integrate with all major operating systems of their end users.
Citrix also offered a technology called HDX which offered a highly optimized
and adaptive system that allows Citrix to closely emulate the level of
performance users require in graphics intensive applications running on local
hardware. When our end users began to see early versions of our Citrix based
solution, it became clear this was the direction our team should pursue.

After meeting with some Citrix engineers, we learned that that Citrix offered a
cloud based solution in which they would host the bulk of the administrative
infrastructure required to deploy their products. Their cloud solution could
be configured to deploy and manage our VDI in AWS and had the flexibility to
utilize a hybrid model that could deploy resources in a variety of other local
and cloud based scenarios.


\section{Lessons Learned}

Windows based software presented some unique challenges to our team. Our team
had to embrace commercial off the shelf hardware due to the limited nature of
open source software in the Windows environment. Many HUBzero components
heavily integrated with open source tools, and the lack of open source tools in
many Windows environments forced our team to rethink how we approached our
development. If a piece of software didn't do what we needed, we
couldn't fork the software and add our missing feature, we had to look at
solving problems from different angles.

Our Linux tool sharing platform concentrated on development and collaboration
in constructing applications, we found our Windows users were less concerned
with developing the software themselves, they often preferred to purchase
software instead. This was not problem, however this realization helped us
strip out some features that we largely unneeded in this particular
environment.

Many Microsoft WIndows tools are licensed tools, often requiring a great deal
of technical and legal setup before the tools could be utilized. This setup was
far more involved and time consuming than we initially predicted. We also had
to overcome our staff's limited Microsoft Windows infrastructure
experience. Over the lifetime of HUBzero, we built a staff based on Linux
skillsets, shifting to Windows proved difficult, not only were we missing
skills, but we often dealt with staff that even had ideological issues with
Microsoft Windows itself.

Client interaction over the internet with remote environments can be
challenging. This was nothing new to us, we struggled with the client
interaction even in our Linux based system, but the importance of this with our
Windows users was even more pronounced. Missing features such as clipboard cut
and paste meant a great deal of lost productivity to our users. Local drive
access was a critical missing feature from our early efforts. We tended to
downplay missing features such as these because we were able to overcome these
limitations ourselves in a variety of ways, but as we expanded our user base, a
growing number of users found these missing features more and more important.


\section{Conclusions}

Years of developing technology on Linux based platforms taught our entire team to look at problems from a certain perspective. Even many of our users were proficient Linux users, so the solutions we devise often assume a lot about our environment and our users. As we begin to look at integrating Microsoft WIndows into HUBzero, we realized we needed to revisit they way we approached designing and developing our system. 

The core of our system will likely remain Linux, but we can no longer construct interfaces, APIs, and packages under the assumption our product will work in a homogenous environment. A key component of HUBzero has been collaboration, but we realized there is a new dimension of collaboration emerging. Early versions of HUBzero focused on collaboration in the development of applications, but we largely missed collaboration in the actual use of the tools and the workflow of how multiple tools worked together to solve complex problems. As we continue our efforts to support Microsoft Windows





% references section

% can use a bibliography generated by BibTeX as a .bbl file
% BibTeX documentation can be easily obtained at:
% http://mirror.ctan.org/biblio/bibtex/contrib/doc/
% The IEEEtran BibTeX style support page is at:
% http://www.michaelshell.org/tex/ieeetran/bibtex/
%\bibliographystyle{IEEEtran}
% argument is your BibTeX string definitions and bibliography database(s)
%\bibliography{IEEEabrv,../bib/paper}
%
% <OR> manually copy in the resultant .bbl file
% set second argument of \begin to the number of references
% (used to reserve space for the reference number labels box)

\begin{thebibliography}{1}

\bibitem{TECHCON:Kapadia96}
Kapadia, Nirav H., Mark S. Lundstrom, and Jose AB Fortes. \emph{A network-based simulation laboratory for collaborative research and technology transfer.} Semiconductor Research Corporation's TECHCON 96 (1996).

\bibitem{ETechNano:Lundstrom06}
Lundstrom, Mark, and Gerhard Klimeck. \emph{The NCN: science, simulation, and cyber services.} Emerging Technologies-Nanoelectronics, 2006 IEEE Conference on. IEEE, 2006.

\bibitem{Rappture:McLennan04}
M. McLennan, �\emph{The Rappture Toolkit}, http://rappture.org (2004)

\end{thebibliography}


\end{document}


